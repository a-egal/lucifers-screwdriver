\documentclass[12pt]{article}

\author{Daniel Wysocki}
\title{
  Lucifer's Screwdriver:
  A machine learning approach to predicting potentially hazardous asteroids
}
\date{April 24, 2016}

\usepackage[margin=1in]{geometry}

\usepackage{graphicx}

\usepackage{amsmath,amssymb,nicefrac,mathtools,physics,siunitx}

\usepackage{hyperref}

\usepackage{caption,subcaption}
\usepackage{enumerate}

\usepackage{listings,verbatim}
\usepackage{color,soul}

\usepackage[style=trad-plain,citestyle=authoryear,maxcitenames=3]{biblatex}
\addbibresource{bibliography.bib}

\DeclareMathOperator{\MOID}{MOID}


\begin{document}

\maketitle

\begin{abstract}



\end{abstract}


\section{Introduction}
\label{sec:intro}



\section{Data}
\label{sec:data}

Orbital and photometric data were obtained from the Minor Planet Center.



\section{Simulations}
\label{sec:sim}

Minimum orbit intersection distances (MOID) were obtained by the standard Fortran package written by Giovanni Gronchi. Objects which satisfy $\MOID < \SI{0.05}{AU}$ and $H > 22$ are labeled as potentially hazardous objects (PHO).



\section{Predictive Model}
\label{sec:model}

We train a deep neural network on the orbital parameters, with their PHO status as labels. The result is a predictive model, which, given orbital parameters and absolute magnitude, can predict an object's PHO status.



\section{Results}
\label{sec:results}





\section{Conclusions}
\label{sec:conclusions}




\end{document}
